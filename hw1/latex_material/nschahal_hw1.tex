\documentclass[12pt]{article}

\usepackage{amsmath}
\usepackage{amssymb}
\usepackage{graphicx}
\usepackage{float} % for [H] exact placement of floats

% --- ADDED: table styling packages & colors ---
\usepackage{booktabs}
\usepackage[table]{xcolor}
\usepackage{tabularx}  % auto-stretch columns to width
\usepackage{array}
\usepackage{listings} % For code snippets
% Listings configuration for code blocks
\lstset{
    language=C++,
    basicstyle=\ttfamily\small,
    frame=single,
    captionpos=b,
    breaklines=true,
    breakatwhitespace=true,
    numbers=left,
    numberstyle=\tiny\color{gray},
    xleftmargin=1em,
    tabsize=2,
    mathescape=false
}



\definecolor{tbl-hdr}{HTML}{0F172A}   % header background
\definecolor{tbl-rowA}{HTML}{0B1020}  % zebra A
\definecolor{tbl-rowB}{HTML}{141B2C}  % zebra B
\definecolor{tbl-accent}{HTML}{4CC3FF}% header underline
\definecolor{tbl-text}{HTML}{DCE7FF}  % light text
\newcolumntype{L}[1]{>{\raggedright\arraybackslash}p{#1}}
\newcolumntype{C}[1]{>{\centering\arraybackslash}p{#1}}
% --- END ADDED ---

\usepackage{hyperref}
\usepackage{algorithm}
\usepackage{algorithmicx}
\usepackage{algpseudocode}
\usepackage{fancyhdr}
\usepackage{pdflscape}
\usepackage{pdfpages}

\title{CIS 5800 Machine Perception Homework-1 \\ University of Pennsylvania}
\author{Navjot Singh Chahal \\ \href{mailto:nschahal@seas.upenn.edu}{nschahal@seas.upenn.edu}}
\date{\today}

\pagestyle{fancy}
\fancyhf{}
\fancyhead[C]{CIS 5800 Machine Perception HW1 University of Pennsylvania}
\fancyfoot[R]{\thepage}
\fancyfoot[L]{Navjot Singh Chahal}

\begin{document}

\maketitle

\section*{1.1}
Using the perspective projection equations:
\[
x = f \times \frac{X}{Z}
\qquad
y = f \times \frac{Y}{Z}
\]

Where:
\begin{itemize}
    \item $(X, Y, Z) = (300, 600, 1200)$ are the camera coordinates
    \item $(x, y) = (55, 110)$ are the image coordinates
\end{itemize}

Solving for focal length $f$:

From the $x$-coordinate equation:
\[
f = x \times \frac{Z}{X} = 55 \times \frac{1200}{300} = 55 \times 4 = 220
\]

From the $y$-coordinate equation:
\[
f = y \times \frac{Z}{Y} = 110 \times \frac{1200}{600} = 110 \times 2 = 220
\]

Both equations give the same result, confirming that $f = 220$ pixels.

\section*{1.2}
Yes, you can determine whether the tree will hit the projection center (camera) when it falls, even without knowing the tree's height.

% \textbf{Proof:}

\begin{itemize}
    \item Tree bottom projects to $B = (0, y_1)$
    \item Tree top projects to $T = (0, y_2)$
    \item Focal length $f = 1$
    \item Both points have $x$-coordinate $= 0$, so the tree lies on the optical axis plane x=0.
    \item $Y_\text{bottom}$ and $Y_\text{top}$ are the actual heights of the tree's bottom and top in camera coordinates.
    \item $Z_\text{tree}$ is the distance from the camera to the tree along the Z-axis.
    \item Tree height: $h = Y_\text{top} - Y_\text{bottom}$
    \item Distance from tree base to camera: $d = \sqrt{Y_\text{bottom}^2 + Z_\text{tree}^2}$
    \item The tree will hit the camera if its height $h$ is greater than or equal to the distance $d$.
\end{itemize}

% \textbf{Mathematical Analysis:}

From perspective projection (with $f = 1$ and $x = 0$ for both points):

\begin{align*}
    \text{Bottom:} \quad & y_1 = \frac{Y_\text{bottom}}{Z_\text{tree}} \implies Y_\text{bottom} = y_1 Z_\text{tree} \\
    \text{Top:} \quad & y_2 = \frac{Y_\text{top}}{Z_\text{tree}} \implies Y_\text{top} = y_2 Z_\text{tree}
\end{align*}

Tree height:
\[
h = Y_\text{top} - Y_\text{bottom} = (y_2 - y_1) Z_\text{tree}
\]

Distance from tree base to camera:
\[
d = \sqrt{Y_\text{bottom}^2 + Z_\text{tree}^2} = Z_\text{tree} \sqrt{y_1^2 + 1}
\]

Collision condition: The tree hits the camera if its height $\geq$ distance to camera:
\[
h \geq d \implies (y_2 - y_1) Z_\text{tree} \geq Z_\text{tree} \sqrt{y_1^2 + 1}
\]
\[
y_2 - y_1 \geq \sqrt{y_1^2 + 1}
\]

\textbf{Hence!}

The tree will hit the camera if and only if:
\[
y_2 - y_1 \geq \sqrt{y_1^2 + 1}
\]

\section*{2.1}
% \textbf{Homogeneous coordinates and projective geometry:}

\begin{itemize}
    \item If $Z \neq 0$: $[X, Y, Z]$ represents an affine point.
    \[
    (x, y) = \left( \frac{X}{Z}, \frac{Y}{Z} \right)
    \]
    \item If $Z = 0$: $[X, Y, 0]$ represents a point at infinity in the direction $(X, Y)$.
    \begin{itemize}
        \item All nonzero scalar multiples are the same projective point.
        \item $[0, 0, 0]$ is not a valid homogeneous coordinate.
    \end{itemize}
    \item Points at infinity correspond to where parallel lines meet (vanishing points in computer vision).
\end{itemize}

\section*{2.2}
For lines through pairs of points in $\mathbb{P}^2$, we use the cross product formula: If $P_1 = [x_1, y_1, z_1]$ and $P_2 = [x_2, y_2, z_2]$, then the line $L = P_1 \times P_2 = [y_1z_2 - z_1y_2,\; z_1x_2 - x_1z_2,\; x_1y_2 - y_1x_2]$.

\begin{enumerate}
    \item[(a)] Points: $[-2,\, 5,\, 3]$ and $[1,\, 3,\, 4]$
    \begin{align*}
        \text{First component:} &\quad 5 \times 4 - 3 \times 3 = 20 - 9 = 11 \\
        \text{Second component:} &\quad 3 \times 1 - (-2) \times 4 = 3 + 8 = 11 \\
        \text{Third component:} &\quad (-2) \times 3 - 5 \times 1 = -6 - 5 = -11
    \end{align*}
    Line equation: $11x + 11y - 11z = 0$ \\
     $x + y - z = 0$

    \item[(b)] Points: $[a,\, 0,\, b]$ and $[0,\, c,\, b]$
    \begin{align*}
        \text{First component:} &\quad 0 \times b - b \times c = -bc \\
        \text{Second component:} &\quad b \times 0 - a \times b = -ab \\
        \text{Third component:} &\quad a \times c - 0 \times 0 = ac
    \end{align*}
    Line equation: $-bcx - aby + acz = 0$ \\
    Or: $bcx + aby - acz = 0$

    \item[(c)] Points: $[a,\, 0,\, 0]$ and $[0,\, 0,\, a]$
    \begin{align*}
        \text{First component:} &\quad 0 \times a - 0 \times 0 = 0 \\
        \text{Second component:} &\quad 0 \times 0 - a \times a = -a^2 \\
        \text{Third component:} &\quad a \times 0 - 0 \times 0 = 0
    \end{align*}
    Line equation: $-a^2y = 0$ \\
    Or: $y = 0$
\end{enumerate}

\section*{2.3}

\subsection*{(a) $3x - y + 2w = 0$ and $x + 5y - w = 0$}

Line vectors,
\begin{align*}
    \mathbf{l}_1 &= [3, -1, 2]^T \\
    \mathbf{l}_2 &= [1, 5, -1]^T
\end{align*}
Next, we compute their cross product to find the intersection point $\mathbf{p}$:
\begin{align*}
    \mathbf{p} &= \mathbf{l}_1 \times \mathbf{l}_2 = 
    \begin{vmatrix}
        \mathbf{i} & \mathbf{j} & \mathbf{k} \\
        3 & -1 & 2 \\
        1 & 5 & -1
    \end{vmatrix} \\
    &= \mathbf{i}((-1)(-1) - (2)(5)) - \mathbf{j}((3)(-1) - (2)(1)) + \mathbf{k}((3)(5) - (-1)(1)) \\
    &= \mathbf{i}(1 - 10) - \mathbf{j}(-3 - 2) + \mathbf{k}(15 + 1) \\
    &= [-9, 5, 16]^T
\end{align*}
The point of intersection is $\mathbf{[-9, 5, 16]}$.

\subsection*{(b) $2x - 6w = 0$ and $5x - 2y = 0$}

\begin{align*}
    2x + 0y - 6w = 0 &\implies \mathbf{l}_1 = [2, 0, -6]^T, \text{ which can be simplified to } [1, 0, -3]^T. \\
    5x - 2y + 0w = 0 &\implies \mathbf{l}_2 = [5, -2, 0]^T.
\end{align*}
Now, we compute the cross product using the simplified vector for $\mathbf{l}_1$:
\begin{align*}
    \mathbf{p} &= \mathbf{l}_1 \times \mathbf{l}_2 = 
    \begin{vmatrix}
        \mathbf{i} & \mathbf{j} & \mathbf{k} \\
        1 & 0 & -3 \\
        5 & -2 & 0
    \end{vmatrix} \\
    &= \mathbf{i}((0)(0) - (-3)(-2)) - \mathbf{j}((1)(0) - (-3)(5)) + \mathbf{k}((1)(-2) - (0)(5)) \\
    &= \mathbf{i}(0 - 6) - \mathbf{j}(0 + 15) + \mathbf{k}(-2 - 0) \\
    &= [-6, -15, -2]^T
\end{align*}
We can simplify this point by multiplying by -1. The point of intersection is $\mathbf{[6, 15, 2]}$.

\subsection*{(c) $7x + y - w = 0$ and $w = 0$}

\begin{align*}
    \mathbf{l}_1 &= [7, 1, -1]^T \\
    \mathbf{l}_2 &= [0, 0, 1]^T
\end{align*}
The line $w=0$ is the \textbf{line at infinity}. The intersection of any finite line with the line at infinity gives the "direction" of that line, represented as a point at infinity.
Let's compute the cross product:
\begin{align*}
    \mathbf{p} &= \mathbf{l}_1 \times \mathbf{l}_2 = 
    \begin{vmatrix}
        \mathbf{i} & \mathbf{j} & \mathbf{k} \\
        7 & 1 & -1 \\
        0 & 0 & 1
    \end{vmatrix} \\
    &= \mathbf{i}((1)(1) - (-1)(0)) - \mathbf{j}((7)(1) - (-1)(0)) + \mathbf{k}((7)(0) - (1)(0)) \\
    &= \mathbf{i}(1 - 0) - \mathbf{j}(7 - 0) + \mathbf{k}(0 - 0) \\
    &= [1, -7, 0]^T
\end{align*}
The point of intersection is $\mathbf{[1, -7, 0]}$, which is a point at infinity as expected.

\section*{2.4}
Given lines:

\[
\ell_1: w = 0 \implies [0,\, 0,\, 1]
\]
\[
\ell_2: x + 2y + w = 0 \implies [1,\, 2,\, 1]
\]

Intersection point of given two lines:
\[
\ell_1 \cap \ell_2 = [0,\, 0,\, 1] \times [1,\, 2,\, 1] = [-2,\, 1,\, 0]
\]
This is a point at infinity: $[-2,\, 1,\, 0]$.

Now, checking if this point lies on each $\ell_3$:

\begin{enumerate}
    \item[(a)] $\ell_3: x + 2y + 6w = 0 \implies [1,\, 2,\, 6]$
    
    Dot product: $[-2,\, 1,\, 0] \cdot [1,\, 2,\, 6] = -2 + 2 + 0 = 0$ \\
    \textbf{Yes}, $[-2,\, 1,\, 0]$ lies on $\ell_3$.

    \item[(b)] $\ell_3: -3x - 6y + 6w = 0 \implies [-3,\, -6,\, 6]$
    
    Dot product: $[-2,\, 1,\, 0] \cdot [-3,\, -6,\, 6] = 6 - 6 + 0 = 0$ \\
    \textbf{Yes}, $[-2,\, 1,\, 0]$ lies on $\ell_3$.

    \item[(c)] $\ell_3: 7x + y - w = 0 \implies [7,\, 1,\, -1]$
    
    Dot product: $[-2,\, 1,\, 0] \cdot [7,\, 1,\, -1] = -14 + 1 + 0 = -13 \neq 0$ \\
    \textbf{No}, $[-2,\, 1,\, 0]$ does not lie on $\ell_3$.
\end{enumerate}

\section*{2.5}

Let $T$ be a $3 \times 3$ projective transformation matrix. The action of $T$ on the standard basis vectors gives its columns:
\[
T \times [1,\,0,\,0]^T = \text{first column of } T = [1,\,-1,\,1]^T
\]
\[
T \times [0,\,1,\,0]^T = \text{second column of } T = [1,\,-2,\,2]^T
\]
\[
T \times [0,\,0,\,1]^T = \text{third column of } T = [-1,\,2,\,-1]^T
\]

Thus, the projective transformation $T$ is:
\[
T = \begin{pmatrix}
1 & 1 & -1 \\
-1 & -2 & 2 \\
1 & 2 & -1
\end{pmatrix}
\]

More generally, the parametric family of projective transformations is:
\[
T(\lambda_1,\,\lambda_2,\,\lambda_3) = \begin{pmatrix}
\lambda_1 & \lambda_2 & -\lambda_3 \\
-\lambda_1 & -2\lambda_2 & 2\lambda_3 \\
\lambda_1 & 2\lambda_2 & -\lambda_3
\end{pmatrix}
\]
where $\lambda_1,\,\lambda_2,\,\lambda_3$ are any nonzero real numbers such that $\det(T) \neq 0$.

\textbf{Valid solutions:}
\begin{itemize}
    \item $\lambda_1 = 1,\ \lambda_2 = 1,\ \lambda_3 = 1$ \hspace{1em} (shown above)
    \item $\lambda_1 = 2,\ \lambda_2 = 1,\ \lambda_3 = 1$
    \item $\lambda_1 = 1,\ \lambda_2 = -1,\ \lambda_3 = 3$
\end{itemize}

\section*{2.6}


\begin{enumerate}
    \item \textbf{Horizontal lines remain horizontal}

    In homogeneous coordinates, all horizontal lines (e.g., $y = c$) intersect at the same point at infinity, which has coordinates $\mathbf{p}_h = [1,\,0,\,0]^T$. If horizontal lines are mapped to horizontal lines, this point must be mapped to itself (or a scaled version of itself).

    Let $H = \begin{pmatrix}
        h_{11} & h_{12} & h_{13} \\
        h_{21} & h_{22} & h_{23} \\
        h_{31} & h_{32} & h_{33}
    \end{pmatrix}$.

    The mapping is $H \mathbf{p}_h \propto \mathbf{p}_h$:
    \[
    H \begin{pmatrix} 1 \\ 0 \\ 0 \end{pmatrix} = \begin{pmatrix} h_{11} \\ h_{21} \\ h_{31} \end{pmatrix} \propto \begin{pmatrix} 1 \\ 0 \\ 0 \end{pmatrix}
    \]
    This implies $h_{21} = 0$ and $h_{31} = 0$.

    \item \textbf{Vertical lines map to lines through $(0, 3)$}

    All vertical lines ($x = c$) intersect at the point at infinity $\mathbf{p}_v = [0,\,1,\,0]^T$. The transformation maps this point to a new location. Since all the resulting lines pass through the point $(0, 3)$, the image of $\mathbf{p}_v$ must be the point $(0, 3)$. The homogeneous coordinates for $(0, 3)$ are $[0,\,3,\,1]^T$.

    The mapping is $H \mathbf{p}_v \propto [0,\,3,\,1]^T$:
    \[
    H \begin{pmatrix} 0 \\ 1 \\ 0 \end{pmatrix} = \begin{pmatrix} h_{12} \\ h_{22} \\ h_{32} \end{pmatrix} \propto \begin{pmatrix} 0 \\ 3 \\ 1 \end{pmatrix}
    \]
    This implies $h_{12} = 0$ and the ratio $h_{22} : h_{32} = 3 : 1$. So, $h_{22} = 3k$, $h_{32} = k$ for some scalar $k$.

    \item \textbf{The point $(0, 0)$ is a fixed point}

    The point $(0, 0)$ is $[0,\,0,\,1]^T$ in homogeneous coordinates. A fixed point is mapped to itself.

    The mapping is $H [0,\,0,\,1]^T \propto [0,\,0,\,1]^T$:
    \[
    H \begin{pmatrix} 0 \\ 0 \\ 1 \end{pmatrix} = \begin{pmatrix} h_{13} \\ h_{23} \\ h_{33} \end{pmatrix} \propto \begin{pmatrix} 0 \\ 0 \\ 1 \end{pmatrix}
    \]
    This implies $h_{13} = 0$, $h_{23} = 0$. Let $h_{33} = m$.

    So far, our matrix looks like:
    \[
    H = \begin{pmatrix}
        h_{11} & 0 & 0 \\
        0 & 3k & 0 \\
        0 & k & m
    \end{pmatrix}
    \]

    \item \textbf{The point $(1, 1)$ is a fixed point}

    The point $(1, 1)$ is $[1,\,1,\,1]^T$ in homogeneous coordinates. This point is also fixed.

    The mapping is $H [1,\,1,\,1]^T \propto [1,\,1,\,1]^T$:
    \[
    H \begin{pmatrix} 1 \\ 1 \\ 1 \end{pmatrix} = \begin{pmatrix} h_{11} \\ 3k \\ k + m \end{pmatrix} \propto \begin{pmatrix} 1 \\ 1 \\ 1 \end{pmatrix}
    \]
    For the output vector to be proportional to $[1,\,1,\,1]^T$, its components must be equal:
    \begin{align*}
        h_{11} &= 3k \\
        3k &= k + m \implies m = 2k
    \end{align*}

    Substituting back, the matrix is:
    \[
    H = \begin{pmatrix}
        3k & 0 & 0 \\
        0 & 3k & 0 \\
        0 & k & 2k
    \end{pmatrix}
    = k \begin{pmatrix}
        3 & 0 & 0 \\
        0 & 3 & 0 \\
        0 & 1 & 2
    \end{pmatrix}
    \]
    Since a homography is defined up to scale, we can set $k = 1$:
    \[
    \boxed{
        H = \begin{pmatrix}
            3 & 0 & 0 \\
            0 & 3 & 0 \\
            0 & 1 & 2
        \end{pmatrix}
    }
    \]
\end{enumerate}

\section*{2.7}
    Let $H$ be a general $3 \times 3$ homography matrix and $p$ a generic point on the line at infinity:
    \[
    H = \begin{pmatrix}
    h_{11} & h_{12} & h_{13} \\
    h_{21} & h_{22} & h_{23} \\
    h_{31} & h_{32} & h_{33}
    \end{pmatrix}, \qquad
    p = \begin{pmatrix} x \\ y \\ 0 \end{pmatrix}
    \]

    Transforming $p$ by $H$:
    \[
    p' = H p = \begin{pmatrix}
    h_{11}x + h_{12}y \\
    h_{21}x + h_{22}y \\
    h_{31}x + h_{32}y
    \end{pmatrix}
    \]

    For $p'$ to also lie on the line at infinity, its third component must be zero for all $x, y$:
    \[
    h_{31}x + h_{32}y = 0 \quad \forall\, x, y
    \]
    This is only possible if $h_{31} = 0$ and $h_{32} = 0$.

    Thus, the matrix for an affine transformation has the form:
    \[
    H_{\text{affine}} = \begin{pmatrix}
    h_{11} & h_{12} & h_{13} \\
    h_{21} & h_{22} & h_{23} \\
    0      & 0      & h_{33}
    \end{pmatrix}
    \]

    This structure ensures that finite points remain finite and points at infinity remain at infinity.

    \section*{3.1}
    Given points $E(-1, -1)$ and $F(0, 2)$:

    Slope:
    \[
    m = \frac{2 - (-1)}{0 - (-1)} = \frac{3}{1} = 3
    \]

    Line equation:
    \[
    y = 3x + 2 \implies 3x - y + 2 = 0
    \]

    Coefficients: $[a, b, c] = [3, -1, 2]$

    Given points $G(2, 2)$ and $H(3, -1)$:

    Slope:
    \[
    m = \frac{-1 - 2}{3 - 2} = \frac{-3}{1} = -3
    \]

    Line equation:
    \[
    y = -3x + 8 \implies 3x + y - 8 = 0
    \]

    Coefficients: $[a, b, c] = [3, 1, -8]$

    \section*{3.2}
    From 3.1 we have:

    EF: $3x - y + 2 = 0$ \\
    GH: $3x + y - 8 = 0$

    Intersection of $EF \cap GH$:

    Solving:
    \[
    \begin{cases}
    3x - y + 2 = 0 \\
    3x + y - 8 = 0
    \end{cases}
    \]
    Add the equations:
    \[
    (3x - y + 2) + (3x + y - 8) = 0 \implies 6x - 6 = 0 \implies x = 1
    \]
    Substitute $x = 1$ into $3x - y + 2 = 0$:
    \[
    3(1) - y + 2 = 0 \implies 5 - y = 0 \implies y = 5
    \]
    So the intersection is $(1,\,5)$.
    \\

    Coefficients of Lines $EG$ and $FH$:

    $E(-1, -1),\ G(2, 2)$: Slope $m = \frac{2 - (-1)}{2 - (-1)} = 1$ \\
    Equation: $y = x$ or $x - y = 0$ (coefficients $[1,\, -1,\, 0]$)

    $F(0, 2),\ H(3, -1)$: Slope $m = \frac{-1 - 2}{3 - 0} = -1$ \\
    Equation: $y = -x + 2$ or $x + y - 2 = 0$ (coefficients $[1,\, 1,\, -2]$)
    \\

    Intersection $EG \cap FH$:

    Solving:
    \[
    \begin{cases}
    x - y = 0 \\
    x + y - 2 = 0
    \end{cases}
    \]
    From $x - y = 0$, $x = y$. Substitute into $x + y - 2 = 0$:
    \[
    x + x - 2 = 0 \implies 2x = 2 \implies x = 1,\ y = 1
    \]
    So the intersection is $(1,\,1)$.

    \section*{3.3}

    Euclidean coordinates of the points in homogeneous form:
    \begin{align*}
        a = [0,\,0,\,1]^T \rightarrow e = [-1,\,-1,\,1]^T \\
        b = [0,\,2,\,1]^T \rightarrow f = [0,\,2,\,1]^T \\
        c = [2,\,2,\,1]^T \rightarrow g = [2,\,2,\,1]^T \\
        d = [2,\,0,\,1]^T \rightarrow h = [3,\,-1,\,1]^T
    \end{align*}

    The transformation for each point is $x' \sim Mx$, which means $x' = k \cdot Mx$ for some scalar $k$.

    \textbf{Linear Dependency in $\mathbb{P}^2$}

    Since we are in $\mathbb{P}^2$, any four points are linearly dependent. Let's express point $d$ as a combination of $a$, $b$, and $c$:
    \[
    d = c_1 a + c_2 b + c_3 c
    \]
    \[
    \begin{pmatrix} 2 \\ 0 \\ 1 \end{pmatrix}
    = c_1 \begin{pmatrix} 0 \\ 0 \\ 1 \end{pmatrix}
    + c_2 \begin{pmatrix} 0 \\ 2 \\ 1 \end{pmatrix}
    + c_3 \begin{pmatrix} 2 \\ 2 \\ 1 \end{pmatrix}
    \]
    Solving this system gives $c_1 = 1$, $c_2 = -1$, $c_3 = 1$. So, $d = a - b + c$.

    \textbf{Solving for Scaling Factors}

    This relationship must be preserved for the destination points, up to individual scaling factors $k_i$:
    \[
    k_4 h = k_1 e - k_2 f + k_3 g
    \]
    \[
    k_4 \begin{pmatrix} 3 \\ -1 \\ 1 \end{pmatrix}
    = k_1 \begin{pmatrix} -1 \\ -1 \\ 1 \end{pmatrix}
    - k_2 \begin{pmatrix} 0 \\ 2 \\ 1 \end{pmatrix}
    + k_3 \begin{pmatrix} 2 \\ 2 \\ 1 \end{pmatrix}
    \]
    Since the overall transformation has a scale ambiguity, we can fix one scalar, for instance, $k_4 = 1$. This gives a system of three linear equations:
    \begin{align*}
        3 &= -k_1 + 2k_3 \\
        -1 &= -k_1 - 2k_2 + 2k_3 \\
        1 &= k_1 - k_2 + k_3
    \end{align*}
    Solving this system yields:
    \[
    k_1 = 1,\quad k_2 = 2,\quad k_3 = 2
    \]

    \textbf{Constructing the Matrix $M$}

    Now we have three definitive mappings:
    \begin{align*}
        M a = k_1 e = [-1,\,-1,\,1]^T \\
        M b = k_2 f = [0,\,4,\,2]^T \\
        M c = k_3 g = [4,\,4,\,2]^T
    \end{align*}

    We can write this as a single matrix equation: $M [a\ b\ c] = [k_1 e\ k_2 f\ k_3 g]$.

    Let
    \[
    P_{\text{src}} = \begin{pmatrix}
    0 & 0 & 2 \\
    0 & 2 & 2 \\
    1 & 1 & 1
    \end{pmatrix}, \qquad
    P_{\text{dst}} = \begin{pmatrix}
    -1 & 0 & 4 \\
    -1 & 4 & 4 \\
    1 & 2 & 2
    \end{pmatrix}
    \]
    Then $M = P_{\text{dst}} \cdot (P_{\text{src}})^{-1}$.

    The inverse of $P_{\text{src}}$ is
    \[
    (P_{\text{src}})^{-1} =
    \begin{pmatrix}
    0 & -\frac{1}{2} & \frac{1}{2} \\
    -\frac{1}{2} & \frac{1}{2} & 0 \\
    1 & 0 & 0
    \end{pmatrix}
    \]
    Multiplying the matrices gives:
    \[
    M = P_{\text{dst}} \cdot (P_{\text{src}})^{-1} =
    \begin{pmatrix}
    -1 & 0 & 4 \\
    -1 & 4 & 4 \\
    1 & 2 & 2
    \end{pmatrix}
    \begin{pmatrix}
    0 & -\frac{1}{2} & \frac{1}{2} \\
    -\frac{1}{2} & \frac{1}{2} & 0 \\
    1 & 0 & 0
    \end{pmatrix}
    = \frac{1}{2}
    \begin{pmatrix}
    4 & 1 & -2 \\
    0 & 5 & -2 \\
    0 & 1 & 2
    \end{pmatrix}
    \]   or, up to scale,
    \[
    M \sim 
    \begin{pmatrix}
    4 & 1 & -2 \\
    0 & 5 & -2 \\
    0 & 1 & 2
    \end{pmatrix}
    \]

    \section*{3.4}
    The transformation matrix is
    \[
    M = \begin{pmatrix}
    4 & 1 & -2 \\
    0 & 5 & -2 \\
    0 & 1 & 2
    \end{pmatrix}
    \]

    The coordinates of point $D$ are $(2, 0)$, which in homogeneous form is
    \[
    d = \begin{pmatrix} 2 \\ 0 \\ 1 \end{pmatrix}
    \]
    The coordinates of point $H$ are $(3, -1)$, which in homogeneous form is
    \[
    h = \begin{pmatrix} 3 \\ -1 \\ 1 \end{pmatrix}
    \]

    We multiply the matrix $M$ by the vector $d$:
    \[
    M d = 
    \begin{pmatrix}
    4 & 1 & -2 \\
    0 & 5 & -2 \\
    0 & 1 & 2
    \end{pmatrix}
    \begin{pmatrix} 2 \\ 0 \\ 1 \end{pmatrix}
    =
    \begin{pmatrix}
    4 \times 2 + 1 \times 0 + (-2) \times 1 \\
    0 \times 2 + 5 \times 0 + (-2) \times 1 \\
    0 \times 2 + 1 \times 0 + 2 \times 1
    \end{pmatrix}
    =
    \begin{pmatrix}
    8 - 2 \\
    -2 \\
    2
    \end{pmatrix}
    =
    \begin{pmatrix}
    6 \\ -2 \\ 2
    \end{pmatrix}
    \]

    Check for proportionality:
    \[
    \begin{pmatrix}
    6 \\ -2 \\ 2
    \end{pmatrix}
    = 2 \times
    \begin{pmatrix}
    3 \\ -1 \\ 1
    \end{pmatrix}
    \]
    So $M d$ is a nonzero scalar multiple of the target vector $h$.

    \textbf{Hence!} The transformation $M$ correctly maps $d$ to $h$ up to scale.

    \section*{4.1}
    \textbf{Vanishing Points and Vanishing Line}

    Horizontal lines converge at the vanishing point $V_x = (-b, 0)$ and vertical lines at $V_y = (0, h)$. The line connecting these points is the vanishing line (horizon) for the facade.

    In homogeneous coordinates:
    \[
    v_x = [-b,\, 0,\, 1]^T,\qquad v_y = [0,\, h,\, 1]^T
    \]

    The vanishing line is given by the cross product:
    \[
    l_v = v_x \times v_y = [-h,\, b,\, -bh]^T
    \]
    (up to scale for simplicity, $[h,\ -b,\ bh]^T$).

    

    \textbf{Transformation Constraints}

    We seek a homography $H$ that maps the vanishing line $l_v$ to the line at infinity $l_\infty = [0,\, 0,\, 1]^T$.

    Thus, the last row of $H$ must be proportional to $(h,\ -b,\ bh)$.

    \textbf{Fixed Points}

    The origin $(0, 0)$ is fixed:
    \[
    H \begin{pmatrix} 0 \\ 0 \\ 1 \end{pmatrix} \propto \begin{pmatrix} 0 \\ 0 \\ 1 \end{pmatrix}
    \]
    This requires $h_{13} = h_{23} = 0$.

    The point $(1, 1)$ is also fixed:
    \[
    H \begin{pmatrix} 1 \\ 1 \\ 1 \end{pmatrix} = \begin{pmatrix} h_{11} + h_{12} \\ h_{21} + h_{22} \\ h - b + bh \end{pmatrix} \propto \begin{pmatrix} 1 \\ 1 \\ 1 \end{pmatrix}
    \]
    So $h_{11} + h_{12} = h_{21} + h_{22} = h - b + bh$.

    \textbf{Vanishing Points Mapping}

    $V_x$ maps to $[1,\, 0,\, 0]^T$:
    \[
    H \begin{pmatrix} -b \\ 0 \\ 1 \end{pmatrix} = \begin{pmatrix} -b h_{11} \\ -b h_{21} \\ 0 \end{pmatrix} \propto \begin{pmatrix} 1 \\ 0 \\ 0 \end{pmatrix}
    \]
    $\implies h_{21} = 0$

    $V_y$ maps to $[0,\, 1,\, 0]^T$:
    \[
    H \begin{pmatrix} 0 \\ h \\ 1 \end{pmatrix} = \begin{pmatrix} h h_{12} \\ h h_{22} \\ 0 \end{pmatrix} \propto \begin{pmatrix} 0 \\ 1 \\ 0 \end{pmatrix}
    \]
    $\implies h_{12} = 0$

    \textbf{Final Matrix}

    With $h_{12} = h_{21} = h_{13} = h_{23} = 0$, and $h_{11} = h_{22} = h - b + bh$, the homography is:
    \[
    H = \begin{pmatrix}
    h - b + bh & 0 & 0 \\
    0 & h - b + bh & 0 \\
    h & -b & bh
    \end{pmatrix}
    \]
    (up to scale for simplicity, $[h,\ -b,\ bh]^T$).

    \section*{4.2}
    \textbf{Horizon = line through the two vanishing points.}

    Vanishing points: 
    \[
    V_x = (-b,\, 0,\, 1), \qquad V_y = (0,\, h,\, 1)
    \]
    Line through them (in homogeneous form) is 
    \[
    \ell = V_x \times V_y
    \]

    Compute the cross product:
    \[
    \ell = 
    \begin{vmatrix}
    \mathbf{i} & \mathbf{j} & \mathbf{k} \\
    -b & 0 & 1 \\
    0 & h & 1
    \end{vmatrix}
    = [-h,\, b,\, -bh]
    \]
    (up to scale, this is equivalent to $[h,\ -b,\ bh]$).

    \[
        \ell = h x - b y + b h  = 0
    \]

\section*{5.1 Bonus: Affine Transforms (15pts)}

% \textbf{Problem:} Prove that, in the case of affine transformations in the projective plane, it is possible to transform a circle into an ellipse, whereas it is impossible to transform an ellipse into a hyperbola or a parabola.

\subsection*{Conic Sections and the Line at Infinity}

In projective geometry, conic sections are distinguished by their intersection with the line at infinity $\ell_\infty$:
%add reference source Multiple View Geometry in Computer Vision Chapter 2 below


\begin{itemize}
    \item \textbf{Ellipse (including circles):} Does not intersect $\ell_\infty$ at real points
    \item \textbf{Parabola:} Intersects $\ell_\infty$ at exactly one real point (tangent)
    \item \textbf{Hyperbola:} Intersects $\ell_\infty$ at exactly two distinct real points
\end{itemize}

\subsection*{Key Property of Affine Transformations}

From Problem 2.7, we established that affine transformations have the form:
\[
H = \begin{pmatrix}
h_{11} & h_{12} & h_{13} \\
h_{21} & h_{22} & h_{23} \\
0 & 0 & h_{33}
\end{pmatrix}
\]

The crucial property is that \textbf{affine transformations preserve the line at infinity}. 

\textbf{Proof:} Points at infinity $[x, y, 0]$ transform as:
\[
H \begin{pmatrix} x \\ y \\ 0 \end{pmatrix} = \begin{pmatrix} h_{11}x + h_{12}y \\ h_{21}x + h_{22}y \\ 0 \end{pmatrix}
\]
The third coordinate remains 0, so points at infinity map to points at infinity.

\subsection*{Circle $\rightarrow$ Ellipse is Possible}

\textbf{Claim:} Any circle can be transformed into any ellipse via an affine transformation.

\textbf{Proof:}
\begin{enumerate}
    \item A circle has equation $x^2 + y^2 = r^2$ (centered at origin for simplicity)
    \item Consider the affine transformation:
    \[
    H = \begin{pmatrix}
    a & 0 & 0 \\
    0 & b & 0 \\
    0 & 0 & 1
    \end{pmatrix}
    \]
    where $a, b > 0$ and $a \neq b$.
    
    \item Under this transformation, a point $(x, y)$ maps to $(ax, by)$
    \item The circle $x^2 + y^2 = r^2$ becomes:
    \[
    \left(\frac{x'}{a}\right)^2 + \left(\frac{y'}{b}\right)^2 = r^2
    \]
    \[
    \frac{x'^2}{a^2r^2} + \frac{y'^2}{b^2r^2} = 1
    \]
    
    \item This is an ellipse with semi-axes $ar$ and $br$.
    
    \item \textbf{Key insight:} Both circle and ellipse have no real intersections with $\ell_\infty$, so the transformation preserves this topological property.
\end{enumerate}

\subsection*{Ellipse $\rightarrow$ Hyperbola/Parabola is Impossible}

\textbf{Claim:} No affine transformation can map an ellipse to a hyperbola or parabola.

\textbf{Proof by Topological Invariant:}

\textbf{Intersection Analysis}
\begin{itemize}
    \item \textbf{Ellipse:} The general ellipse $\frac{x^2}{a^2} + \frac{y^2}{b^2} = 1$ in homogeneous coordinates becomes:
    \[
    \frac{x^2}{a^2} + \frac{y^2}{b^2} = z^2
    \]
    At the line at infinity ($z = 0$), we get $\frac{x^2}{a^2} + \frac{y^2}{b^2} = 0$, which has no real solutions.
    
    \item \textbf{Hyperbola:} The hyperbola $\frac{x^2}{a^2} - \frac{y^2}{b^2} = 1$ becomes:
    \[
    \frac{x^2}{a^2} - \frac{y^2}{b^2} = z^2
    \]
    At $z = 0$: $\frac{x^2}{a^2} - \frac{y^2}{b^2} = 0 \Rightarrow y = \pm\frac{b}{a}x$
    
    This gives two distinct real intersection points: $[a:b:0]$ and $[a:-b:0]$.
    
    \item \textbf{Parabola:} The parabola $y = x^2$ becomes $yz = x^2$ in homogeneous form.
    At $z = 0$: $0 = x^2 \Rightarrow x = 0$, giving one real intersection point $[0:1:0]$.
\end{itemize}

\textbf{Invariance Under Affine Transformations}

Since affine transformations preserve the line at infinity, they preserve the number of real intersections between any conic and $\ell_\infty$.

\textbf{Intersection Count Invariant:}
\begin{itemize}
    \item Ellipse: 0 real intersections with $\ell_\infty$
    \item Parabola: 1 real intersection with $\ell_\infty$  
    \item Hyperbola: 2 real intersections with $\ell_\infty$
\end{itemize}

\textbf{Conclusion}

Since the number of real intersections with the line at infinity is preserved under affine transformations:

\begin{itemize}
    \item \textbf{Ellipse $\rightarrow$ Parabola:} Impossible (0 intersections $\not\rightarrow$ 1 intersection)
    \item \textbf{Ellipse $\rightarrow$ Hyperbola:} Impossible (0 intersections $\not\rightarrow$ 2 intersections)
\end{itemize}

\subsection*{Also euclidean classification by Discriminant,}

A conic section $Ax^2 + Bxy + Cy^2 + Dx + Ey + F = 0$ has discriminant $\Delta = B^2 - 4AC$.

\begin{itemize}
    \item $\Delta < 0$: Ellipse (including circles when $A = C$ and $B = 0$)
    \item $\Delta = 0$: Parabola
    \item $\Delta > 0$: Hyperbola
\end{itemize}

Under an affine transformation $H = \begin{pmatrix} a & b & p \\ c & d & q \\ 0 & 0 & 1 \end{pmatrix}$, the discriminant transforms as:
\[
\Delta' = \Delta \cdot (\det(A))^2
\]
where $A = \begin{pmatrix} a & b \\ c & d \end{pmatrix}$ is the linear part.

Since $\det(A) \neq 0$ for invertible transformations, the sign of $\Delta$ is preserved:
\begin{itemize}
    \item $\Delta < 0 \Rightarrow \Delta' < 0$ (ellipse stays ellipse)
    \item $\Delta = 0 \Rightarrow \Delta' = 0$ (parabola stays parabola)  
    \item $\Delta > 0 \Rightarrow \Delta' > 0$ (hyperbola stays hyperbola)
\end{itemize}

\textbf{Conclusion:} Affine transformations preserve the topological type of conic sections. Therefore:
\begin{itemize}
    \item \checkmark\ Circle $\rightarrow$ Ellipse: \textbf{Possible} (both have $\Delta < 0$)
    \item $\times$ Ellipse $\rightarrow$ Hyperbola: \textbf{Impossible} ($\Delta < 0 \not\rightarrow \Delta > 0$)
    \item $\times$ Ellipse $\rightarrow$ Parabola: \textbf{Impossible} ($\Delta < 0 \not\rightarrow \Delta = 0$)
\end{itemize}

\section*{5.2 Bonus: Vanishing Point Computation (10pts)}

% \textbf{Problem:} Consider the one-dimensional projective transformation $y = H_{2 \times 2}x$ between points on a line of the real world and points on the corresponding line of the image, where the points $x$ and $y$ are parameterized and given in homogeneous coordinates of the form $(a, b)$. Here, $H_{2 \times 2}$ takes input $x \in \mathbb{P}^1$ to output $y \in \mathbb{P}^1$. Based on the information provided by the homography matrix $H_{2 \times 2}$, find the vanishing point for the line represented by the said homography.

\subsection*{Given Setup}

We have a $2 \times 2$ homography matrix acting on points in $\mathbb{P}^1$ (the projective line):
\[
H_{2 \times 2} = \begin{pmatrix} h_{11} & h_{12} \\ h_{21} & h_{22} \end{pmatrix}
\]

Points on the projective line are represented in homogeneous coordinates as $[a, b]^T$, where:
\begin{itemize}
    \item When $b \neq 0$: the point corresponds to the affine coordinate $\frac{a}{b}$
    \item When $b = 0$: the point $[a, 0]^T$ represents the point at infinity
\end{itemize}

\subsection*{Vanishing Point}

The \textbf{vanishing point} is where parallel lines in the world appear to meet in the image. In the context of a 1D projective transformation, the vanishing point is the image of the point at infinity from the world line.

The point at infinity in $\mathbb{P}^1$ is represented as $[1, 0]^T$.

\subsection*{Computing the Vanishing Point}

\textbf{1.} Apply the homography to the point at infinity:
\[
y_{\text{vanishing}} = H_{2 \times 2} \begin{pmatrix} 1 \\ 0 \end{pmatrix} = \begin{pmatrix} h_{11} \\ h_{21} \end{pmatrix}
\]

\textbf{2.} Convert to affine coordinates (if possible):

\textbf{Case 1:} If $h_{21} \neq 0$, the vanishing point in affine coordinates is:
\[
\boxed{v = \frac{h_{11}}{h_{21}}}
\]

\textbf{Case 2:} If $h_{21} = 0$ and $h_{11} \neq 0$, the vanishing point is at infinity in the image, represented as $[1, 0]^T$.

\textbf{Case 3:} If $h_{21} = h_{11} = 0$, this would mean the first column of $H$ is zero, making $H$ singular (non-invertible), which is not a valid homography.

\subsection*{Geometric Interpretation}

The homography matrix can be written as:
\[
H_{2 \times 2} = \begin{pmatrix} h_{11} & h_{12} \\ h_{21} & h_{22} \end{pmatrix}
\]

The transformation maps:
\begin{itemize}
    \item World point $[x, 1]^T$ (affine coordinate $x$) to image point $[h_{11}x + h_{12}, h_{21}x + h_{22}]^T$
    \item World point at infinity $[1, 0]^T$ to image point $[h_{11}, h_{21}]^T$ (the vanishing point)
\end{itemize}

\subsection*{General Matrix Form}

If we write the homography as:
\[
H_{2 \times 2} = \begin{pmatrix} a & b \\ c & d \end{pmatrix}
\]

Then the vanishing point is:
\[
\boxed{\text{Vanishing point} = \begin{cases}
\frac{a}{c} & \text{if } c \neq 0 \\
\text{point at infinity} & \text{if } c = 0, a \neq 0 \\
\text{undefined (singular)} & \text{if } c = a = 0
\end{cases}}
\]


\subsection*{Final Answer}

For a $2 \times 2$ homography matrix $H_{2 \times 2} = \begin{pmatrix} h_{11} & h_{12} \\ h_{21} & h_{22} \end{pmatrix}$, the vanishing point is:

\[
\boxed{\text{Vanishing Point} = \frac{h_{11}}{h_{21}} \quad \text{(provided } h_{21} \neq 0\text{)}}
\]

This represents the affine coordinate where the point at infinity from the world line appears in the image line.

\end{document}
